\documentclass[11pt,letterpaper]{article}     % Tipo de documento y otras especificaciones
\usepackage[utf8]{inputenc}                   % Para escribir tildes y eñes
\usepackage[spanish]{babel}                   % Para que los títulos de figuras, tablas y otros estén en español
\usepackage[apaciteclassic]{apacite}
\usepackage{geometry}    
\usepackage{textcomp}
\geometry{left=25mm, right=25mm, top=25mm, bottom=25mm} % Tamaño del área de escritura de la página
\usepackage{amsmath}      % Los paquetes ams son desarrollados por la American Mathematical Society
\usepackage{amsfonts}     % y mejoran la escritura de fórmulas y símbolos matemáticos.
\usepackage{booktabs}
\usepackage{subfig}
\usepackage{amssymb}
\usepackage{graphicx}     % Para insertar gráficas
\usepackage{float}		% Para ubicar las tablas y figuras justo después del texto
\usepackage{pdfpages}
\batchmode
\usepackage{enumerate}
\usepackage{siunitx}
\pagestyle{plain} 
\usepackage{graphics}
\pagenumbering{arabic}
\usepackage{multicol}   % Para varias columnas
\usepackage{multirow}
\usepackage{color}%Paquete para colocar color al texto
%====================Español Venezolano Rápido============================
\renewcommand\tablename{Tabla}
\renewcommand\figurename{Figura}
%\renewcommand\prefacename{Prefacio}
\renewcommand\refname{REFERENCIAS}
%\renewcommand\bibname{REFERENCIAS}
\renewcommand\abstractname{Resumen}
%\renewcommand\chaptername{CAPÍTULO}
\renewcommand\appendixname{Apéndice}
\renewcommand\contentsname{ÍNDICE GENERAL}
\renewcommand\listfigurename{LISTA DE FIGURAS}
\renewcommand\listtablename{LISTA DE TABLAS}
\renewcommand\indexname{Índice Alfabético}
\renewcommand\partname{Parte}
\graphicspath{ {RecursosLab5/} }
%\renewcommand\enclname{Adjunto}
%\renewcommand\ccname{Copia a}
%\renewcommand\headtoname{A}
%\renewcommand\pagename{Página}
%\renewcommand\seename{véase}
%\renewcommand\alsoname{véase también}
%\renewcommand\proofname{Demostración}
%\renewcommand\glossaryname{Glosario}
%===================  Español venezolano =====================


\author{\\Omaña Enderson CI:  24.757.361 \\Raven Guillermo CI: 25.476.227\\Profesor: Crespo Jorge \vspace*{1in}}
\title{Universidad Central de Venezuela\\{ Facultad de Ingeniería\\Escuela de Ingeniería Eléctrica\\ Conversión Electromecánica de la energía\\\vspace*{1.5in} }Laboratorio 5\\TRANSFORMADOR MONOFÁSICO. COMPORTAMIENTO\vspace*{1.35in}}
\date{Caracas, \today}

\begin{document}	% Inicio del documento
\maketitle							% Título
\newpage
\tableofcontents
\newpage
\section{Objetivos}
\begin{itemize}
	\item Trabajar con un elemento importante en los sistemas eléctricos.
	\item Obtener la respuesta de la tensión secundaria del transformador para cargas lineales y no lineales con
	tensión de alimentación constante.
\end{itemize}
\section{Marco Teórico}
\subsection{Carga lineal}
En el caso de una carga resistiva, referenciando los parámetros al secundario se tiene el siguiente diagrama fasorial:
\begin{figure}[H]
	\centering
	\includegraphics[width=0.7\linewidth]{lab5/RecursosLab5/cargaResisVector.jpg}
	\caption{Diagrama Fasorial para carga resistiva}
	\label{fig:cargaresisvector}
\end{figure}
Por lo tanto, se observa que las ecuaciones que rigen su comportamiento serán:
\begin{align}
	\frac{Vp}{a}= (Vs+I2*Req) +j\cdot(Is*Xeq)\\
	\left\|\frac{Vp}{a}\right\| = \sqrt{(Vs+I2*Req)^{2}+(Is*Xeq)^{2}}\\
	\measuredangle \frac{Vp}{a} = \arctan\left(\frac{Is*Xeq}{Vs+I2*Req}\right)
\end{align}
Del diagrama fasorial presente en la figura \ref{fig:cargaresisvector} y las ecuaciones previas se pueden deducir 3 cosas: 
\begin{enumerate}
	\item La corriente se encuentra en fase con la tensión por lo tanto las curvas obtenidas en el osciloscopio deben estarlo.
	\item La corriente varia dependiendo de la carga y esta afecta directamente a la tensión de alimentación en magnitud y fase.
	\item Si la resistencia en la carga disminuye la corriente aumenta lo que hace que la tensión de alimentación aumente y el desfase es mayor, en cambio si la resistencia en la carga aumenta la corriente disminuye lo que hace que la tensión de alimentación disminuya y el desfase es menor.
\end{enumerate}
\section{Lista de instrumentos}
\begin{table}[H]
	\caption{Lista de instrumentos de medición y componentes}
	\centering
	\begin{tabular}{|c|c|}
		\hline 
		Instrumento & Alcance ó especificaciones \\ \hline 
		Reostato &  (0-33) $\Omega$ \\  
		\hline 
		Osciloscopio &  - \\ 
		\hline 
		Transformador &  2 KVA; 2:1 \\  
		\hline 
		Voltímetro de hierro móvil &  (0-150) V\\  
		\hline 
		Amperímetro de hierro móvil &(0-1.2) A ó (0-6) A \\ 
		\hline
		Vatímetro & -	\\
		\hline
		Protecciones AC & 25 A; 380 V\\ 
		\hline
		Protecciones DC & -\\ 
		\hline
		Carga inductiva& (Se utilizo un motor DC) \\ 
		\hline
		Carga resistiva-capacitiva& 33 $\Omega$; 300 $\mu F$ \\ 
		\hline
		Puente rectificador & -  \\
		\hline
		Carga lineal & (200,400,600,800,1K)W  \\
		\hline
		Variac & 7 KVA @ 50 A 
		\\
		\hline 
	\end{tabular} 
\end{table}
\section{Condiciones de ensayo}
Estas son las precauciones y normativas necesarias para realizar el laboratorio de forma segura y efectiva: 
\begin{itemize}
    \item \textbf{Respecto a la  vestimenta:} No usar franelas o camisas manga larga, llevar zapatos de goma y pantalones. No usar collares ni pulseras de metal.
    \item \textbf{Previo a las pruebas:} Hacer primero el montaje antes de energizar, al culminarlo preguntar al profesor si las conexiones son correctas para proceder con las pruebas.
    \item \textbf{Respecto a la comunicación:} Mantener informado sobre cualquier cambio en el montaje al compañero de laboratorio y por sobre todo informar si el circuito se encuentra energizado o no.
    \item \textbf{Respecto al ambiente (COVENIN 3172-95):} El transformador debe estar ubicado en un sitio en el cual no tenga corrientes de aire ni fluctuaciones rápidas de la temperatura ambiente.
	\item \textbf{Respecto a las curvas observadas:} No aceptar como adecuada una curva que este llena de ruido, ya que se puede deber a que algún elemento puede estar actuando como antena, esto originara imprecisión en los resultados.
	\item \textbf{Respecto al numero de mediciones:}
	Realizar al menos 4 mediciones para condiciones distintas.
	\item \textbf{Respecto a la elección de componentes y las conexiones:} Evitar los componentes que puedan funcionar como antenas (como resistencias de shunt de tipo mariposa u algun otro que se encuentre muy expuesto) y cuidar los contactos de cada conexión.
    \item \textbf{Respecto a la manipulación:} En caso de maniobrar el circuito energizado manipular con la mano derecha, buscando mayores probabilidades de sobrevivir en caso de un accidente eléctrico.
\end{itemize}
\section{Procedimiento}
\subsection{Carga Lineal}
\begin{enumerate}
	\item Se armara el circuito propuesto en la figura \ref{diagramaLineal}.
	\item Ajustar el reostato al 0\% (solo si es necesario), 25\%, 50\%, 75\% y 100\% de su capacidad y para cada estado realizar mediciones en tensión y corriente en la carga y en el primario, a su vez almacenar las imágenes que representen las señales respectivas. 
\end{enumerate}
\subsection{Carga No Lineal}
\begin{enumerate}
	\item Se conecta el circuito previsto en la figura \ref{DiagramaNoLineal} conectando una carga inductiva.
	\item Se mide la potencia aparente y la potencia activa consumida por la carga.
	\item Se observa la forma de onda de onda de corriente y tensión en la carga se almacenan las imagenes y se capturan las mismas imagenes pero para el primario.
	\item Se cambia la carga a una capacitiva-resistiva y se realizan las mismas mediciones que en los pasos previos, pero se utiliza el circuito de la figura \ref{DiagramaNoLinealCapacitiva}.
\end{enumerate}
\section{Diagramas}
\begin{figure}[H]
    \centering
    \includegraphics{lab5/RecursosLab5/ModeloAlta.jpg}
    \caption{Modelo del transformador referenciado a alta}
    \label{fig:ModeloTransformadorAlta}
\end{figure}
\begin{figure}[H]
    \centering
    \includegraphics{lab5/RecursosLab5/ModeloBaja.jpg}
    \caption{Modelo del transformador referenciado a naja}
    \label{fig:ModeloTransformadorBaja}
\end{figure}
\begin{figure}[H]
    \centering
    \includegraphics[scale=0.7]{lab5/RecursosLab5/CargaResistiva.jpg}
    \caption{Diagrama de conexión con carga líneal}
    \label{diagramaLineal}
\end{figure}
\begin{figure}[H]
    \centering
    \includegraphics[scale=0.7]{lab5/RecursosLab5/CargaInductiva.jpg}
    \caption{Diagrama de conexión con carga no lineal inductiva}
    \label{DiagramaNoLineal}
\end{figure}
\begin{figure}[H]
    \centering
    \includegraphics[scale=0.7]{lab5/RecursosLab5/CargaCapcitivaResistiva.jpg}
    \caption{Diagrama circuital resistiva capacitiva}
    \label{DiagramaNoLinealCapacitiva}
\end{figure}
\section{Resultados experimentales}
\subsection{Carga lineal}
\subsubsection{Condición en vació}
Se fijo la tensión en el lado de baja (120 $\pm$ 1) V con una corriente de (2.3 $\pm$ 0.1) A; además se obtuvo la forma de onda de la corriente en vació del lado de baja en la Figura \ref{ShuntEnVacio}.
\begin{figure}[H]
    \centering
    \includegraphics[scale=0.2]{lab5/RecursosLab5/shuntEnVacio.jpeg}
    \caption{Corriente en lado de baja. Escala 10 mV y 2 ms }
    \label{ShuntEnVacio}
\end{figure}
\subsubsection{Mediciones en RMS y valores calculados}
En la siguiente figura se aprecian los valores obtenidos por medio de  voltímetros, amperímetros y un vatimetro:
\begin{table}[H]
\centering
\caption{Mediciones en lado de alta y baja para diversas cargas resistivas}
\label{experiencia1}
\begin{tabular}{|c|c|c|c|c|c|}
\hline
\textbf{Carga lineal {[}W{]}} & \textbf{$V_{entrada}$ {[}V{]}} & \textbf{$V_{salida}$ {[}V{]}} & \textbf{$I_{entrada}$ {[}A{]}} & \textbf{$I_{salida}$ {[}A{]}} & \textbf{$P_{salida}$ {[}W{]}} \\ \hline
200                           & 120$\pm$1                      & 230$\pm$2                     & 2,9$\pm$0,1                    & 1,01$\pm$0,05                 & 220$\pm$10                    \\ \hline
400                           & 118$\pm$1                      & 224$\pm$2                     & 4,3$\pm$0,1                    & 1,90$\pm$0,05                 & 410$\pm$10                    \\ \hline
600                           & 118$\pm$1                      & 224$\pm$2                     & 5,9$\pm$0,1                    & 2,55$\pm$0,05                 & 620$\pm$10                    \\ \hline
800                           & 112$\pm$1                      & 220$\pm$2                     & 7,5$\pm$0,1                    & 3,60$\pm$0,05                 & 790$\pm$10                    \\ \hline
1000                          & 110$\pm$1                      & 218$\pm$2                     & 9,0$\pm$0,1                    & 4,40$\pm$0,05                 & 950$\pm$10                    \\ \hline
\end{tabular}
\end{table}
Utilizando los resultados previos, extrayendo $R_{cc}$ y $X_{cc}$ del modelo obtenido en la practica previa se obtiene la tensión $\frac{V_{p}}{a}$ cuya expresión del error es la siguiente:
\begin{equation}
    Copia la formula de la incertidumbre
\end{equation}
Con $\frac{V_{p}}{a}$ y la tensión en el secundario se obtiene la regulación por medio de las siguientes formulas:
\begin{align}
    Reg = \frac{\frac{V_{p}}{a}-V_{salida}}{V_{salida}}\\
    \Delta Reg = Copia la formula del error
\end{align}
Con las expresiones previas se obtuvo el siguiente cuadro:
\begin{table}[H]
\centering
\caption{Regulación respecto a la carga lineal}
\label{Regcuadro}
\begin{tabular}{|c|c|c|}
\hline
\textbf{$\frac{V_{p}}{a}$ {[}V{]}} & \textbf{Regulación} & \textbf{Carga lineal {[}W{]}} \\ \hline
                                   &                     & 220 $\pm$ 10                  \\ \hline
                                   &                     & 410 $\pm$ 10                  \\ \hline
                                   &                     & 620$\pm$10                    \\ \hline
                                   &                     & 790$\pm$10                    \\ \hline
                                   &                     & 950$\pm$10                    \\ \hline
\end{tabular}
\end{table}
Con los valores de regulación obtenidos y la corriente en el lado de alta se comparo con su homologo teórico en la siguiente gráfica:
%Pon la grafica por favor calacula todo lo anterior para mis valores de modelo tambien mi Rcc es 1,43$\pm$0,02 y Xcc=0,87$\pm$0,03
\subsubsection{Corrientes en alta y baja para distintas cargas}
\begin{figure}[H]
    \centering
    \includegraphics[scale=0.3]{lab5/RecursosLab5/shuntBaja200W.jpeg}
    \caption{Corriente en baja para carga de 200 W. Escalas 10 mV y 2 ms}
    \label{shuntBaja200W}
\end{figure}
\begin{figure}[H]
    \centering
    \includegraphics[scale=0.3]{lab5/RecursosLab5/shuntAlta200W.jpeg}
    \caption{Corriente en alta para carga de 200 W. Escalas 10 mV y 2 ms}
    \label{shuntAlta200W}
\end{figure}
\begin{figure}[H]
    \centering
    \includegraphics[scale=0.3]{lab5/RecursosLab5/shuntBaja400W.jpeg}
    \caption{Corriente en baja para carga de 200 W. Escalas 10 mV y 2 ms}
    \label{shuntBaja400W}
\end{figure}
\begin{figure}[H]
    \centering
    \includegraphics[scale=0.3]{lab5/RecursosLab5/shuntAlta400W.jpeg}
    \caption{Corriente en alta para carga de 400 W. Escalas 10 mV y 2 ms}
    \label{shuntAlta400W}
\end{figure}
\begin{figure}[H]
    \centering
    \includegraphics[scale=0.3]{lab5/RecursosLab5/shuntBaja600W.jpeg}
    \caption{Corriente en baja para carga de 600 W. Escalas 20 mV y 2 ms}
    \label{shuntBaja600W}
\end{figure}
\begin{figure}[H]
    \centering
    \includegraphics[scale=0.3]{lab5/RecursosLab5/shuntAlta600W.jpeg}
    \caption{Corriente en alta para carga de 600 W. Escalas 20 mV y 2 ms}
    \label{shuntAlta600W}
\end{figure}
\begin{figure}[H]
    \centering
    \includegraphics[scale=0.3]{lab5/RecursosLab5/shuntBaja800W.jpeg}
    \caption{Corriente en baja para carga de 800 W. Escalas 20 mV y 2 ms}
    \label{shuntBaja800W}
\end{figure}
\begin{figure}[H]
    \centering
    \includegraphics[scale=0.3]{lab5/RecursosLab5/shuntAlta800W.jpeg}
    \caption{Corriente en alta para carga de 800 W. Escalas 20 mV y 2 ms}
    \label{shuntAlta800W}
\end{figure}
\begin{figure}[H]
    \centering
    \includegraphics[scale=0.3]{lab5/RecursosLab5/shuntBaja1kW.jpeg}
    \caption{Corriente en baja para carga de 1 KW. Escalas 20 mV y 2 ms}
    \label{shuntBaja1KW}
\end{figure}
\subsection{Carga no lineal}
Se calcula la potencia aparente de la carga por medio de las siguientes expresiones:
\begin{align}
    S = I\cdot V\\
    \Delta S = V\Delta I + I\Delta V
\end{align}
El factor de potencia:
\begin{align}
    fp = \frac{P(salida)}{S} \\
    \Delta fp = \frac{\Delta P(salida) }{S} + \frac{P(salida)\cdot \Delta S}{S^{2}}
\end{align}
De acuerdo a las mediciones y las expresiones previas se obtiene el siguiente cuadro:
\begin{table}[H]
\centering
\caption{Valores obtenidos para cargas no lineales}
\label{ValoresCNL}
\begin{tabular}{|c|c|c|c|c|c|}
\hline
\textbf{$V_{entrada}$ {[}V{]}} & \textbf{$V_{salida}$ {[}V{]}} & \textbf{$I_{entrada}$ {[}A{]}} & \textbf{$I_{salida}$ {[}A{]}} & \textbf{P (salida) {[}W{]}} & \textbf{Carga} \\ \hline
28$\pm$1                       & 54$\pm$2                      & 3,0$\pm$0,1                    & 1,55$\pm$0,05                 & 70 $\pm$ 10                 & Inductiva      \\ \hline
28$\pm$1                       & 53$\pm$1                      & 3,5$\pm$0,1                    & 1,80$\pm$0,05                 & 80 $\pm$ 10                 & Capacitiva     \\ \hline
\multicolumn{3}{|c|}{\textbf{S {[}VA{]}}}                                                       & \multicolumn{3}{c|}{\textbf{fp}}                                             \\ \hline
Inductiva                      & \multicolumn{2}{c|}{Llenar 1}                                  & Inductiva                     & \multicolumn{2}{c|}{Llenar 3}                \\ \hline
Capacitiva                     & \multicolumn{2}{c|}{Llenar 2}                                  & Capacitiva                    & \multicolumn{2}{c|}{Llenar 4}                \\ \hline
\end{tabular}
\end{table}
Se obtuvieron las siguientes curvas para ambas cargas:
\begin{figure}[H]
    \centering
    \includegraphics[scale=0.3]{lab5/RecursosLab5/shuntInductiva.jpeg}
    \caption{Corriente en baja para carga inductiva. Escala 5 mV y 2 ms}
    \label{shuntInductiva}
\end{figure}
\begin{figure}[H]
    \centering
    \includegraphics[scale=0.3]{lab5/RecursosLab5/shuntResisCapacitiva.jpeg}
    \caption{Corriente en baja para carga resistiva capacitiva. Escala 5 mV y 2 ms}
    \label{shuntResisCapacitiva}
\end{figure}
\section{Resultados Teóricos}
Utilizando el modelo del transformador del diagrama \ref{fig:ModeloTransformadorAlta}, se obtuvo de forma teórica para cargas de (200, 400, 600, 800, 1000) W el siguiente cuadro:
\begin{table}[H]
\centering
\caption{Regulación respecto a la carga lineal}
\label{RegcuadroTEORICO}
\begin{tabular}{|c|c|c|c|}
\hline
\textbf{$\frac{V_{p}}{a}$ {[}V{]}} & \textbf{Regulación} & \textbf{Carga lineal {[}W{]}}  & \textbf{I(carga) {[}A{]}} \\ \hline
                                   &                     & 200    &             \\ \hline
                                   &                     & 400    &             \\ \hline
                                   &                     & 600      &              \\ \hline
                                   &                     & 800    &                \\ \hline
                                   &                     & 1000   &                \\ \hline
\end{tabular}
\end{table}
\section{Análisis de resultados}
\section{Conclusión}
\newpage
\section{Anexos}
\begin{figure}[H]
    \centering
    \includegraphics[scale=0.7, angle=-90]{lab5/RecursosLab5/hojaDeDatos1.jpeg}
    \caption{Hoja de datos 1}
\end{figure}
\begin{figure}[H]
    \centering
    \includegraphics[scale=0.7, angle=-90]{lab5/RecursosLab5/hojaDeDatos2.jpeg}
    \caption{Hoja de datos 2}
\end{figure}
\end{document}
