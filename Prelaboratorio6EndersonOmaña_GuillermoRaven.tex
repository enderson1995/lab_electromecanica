\documentclass[11pt,letterpaper]{article}     % Tipo de documento y otras especificaciones
\usepackage[utf8]{inputenc}                   % Para escribir tildes y eñes
\usepackage[spanish]{babel}                   % Para que los títulos de figuras, tablas y otros estén en español
\usepackage[apaciteclassic]{apacite}
\usepackage{geometry}    
\usepackage{textcomp}
\geometry{left=25mm, right=25mm, top=25mm, bottom=25mm} % Tamaño del área de escritura de la página
\usepackage{amsmath}      % Los paquetes ams son desarrollados por la American Mathematical Society
\usepackage{amsfonts}     % y mejoran la escritura de fórmulas y símbolos matemáticos.
\usepackage{booktabs}
\usepackage{subfig}
\usepackage{amssymb}
\usepackage{graphicx}     % Para insertar gráficas
\usepackage{float}		% Para ubicar las tablas y figuras justo después del texto
\usepackage{pdfpages}
\batchmode
\usepackage{enumerate}
\usepackage{siunitx}
\pagestyle{plain} 
\usepackage{graphics}
\pagenumbering{arabic}
\usepackage{multicol}   % Para varias columnas
\usepackage{multirow}
\usepackage{color}%Paquete para colocar color al texto
%====================Español Venezolano Rápido============================
\renewcommand\tablename{Tabla}
\renewcommand\figurename{Figura}
%\renewcommand\prefacename{Prefacio}
\renewcommand\refname{REFERENCIAS}
%\renewcommand\bibname{REFERENCIAS}
\renewcommand\abstractname{Resumen}
%\renewcommand\chaptername{CAPÍTULO}
\renewcommand\appendixname{Apéndice}
\renewcommand\contentsname{ÍNDICE GENERAL}
\renewcommand\listfigurename{LISTA DE FIGURAS}
\renewcommand\listtablename{LISTA DE TABLAS}
\renewcommand\indexname{Índice Alfabético}
\renewcommand\partname{Parte}

%\renewcommand\enclname{Adjunto}
%\renewcommand\ccname{Copia a}
%\renewcommand\headtoname{A}
%\renewcommand\pagename{Página}
%\renewcommand\seename{véase}
%\renewcommand\alsoname{véase también}
%\renewcommand\proofname{Demostración}
%\renewcommand\glossaryname{Glosario}
%===================  Español venezolano =====================


\author{\\Omaña Enderson CI:  24.757.361 \\Raven Guillermo CI: 25.476.227\\Profesor: Crespo Jorge \vspace*{1in}}
\title{Universidad Central de Venezuela\\{ Facultad de Ingeniería\\Escuela de Ingeniería Eléctrica\\ Conversión Electromecánica de la energía\\\vspace*{1.5in} }Laboratorio 6\\MAQUINAS DE CORRIENTE CONTINUA\vspace*{1.35in}}
\date{Caracas, \today}

\begin{document}	% Inicio del documento
\maketitle							% Título
\newpage
\tableofcontents
\newpage
\section{Objetivos}
\begin{itemize}
	\item Estudiar los diferentes tipos de motores de corriente continua y sus
    aplicaciones en la industria.
	\item Modelar en régimen permanente una máquina de corriente continua.
	\item Determinar las curvas características más importantes de las máquinas de
    corriente continua.
    \item Estudiar los problemas asociados a la conmutación de una máquina de
    corriente continua.
\end{itemize}
\section{Marco Teórico}
\subsection{Curva de saturación o vacio}
En las máquinas de corriente continua, la curva de saturación o magnetización
B=f(H) representa la característica interna de su circuito magnético y de sus
bobinas. Para la medición de esta característica no se mide la inducción B en
función del campo magnético H, sino dos cantidades eléctricas proporcionales a
éstas: la fuerza electromotriz generada en vacío, $E_{0}$ contra la corriente del campo principal de excitación $I_{exc}$, obteniéndose así lo que se conoce como la curva de vacío (durante el ensayo la máquina esta sin carga o en vacío) o de saturación (en la curva se visualiza el codo de saturación del circuito magnético de la máquina).

Es conocido que una máquina de p pares de polos, 2a vías en paralelo, y n
conductores; que gira a la velocidad w, genera una fuerza electromotriz E0, igual
a:
\begin{equation}
    Eo = k\cdot \phi \cdot w
\end{equation}
Si se hace girar la máquina a velocidad constante, a través de cualquier
mecanismo externo (ej.: turbina, o en el laboratorio un motor sincrónico), la fuerza electromotriz es directamente proporcional al flujo $\phi$, que a su vez es proporcional a la densidad de flujo magnético (inducción) B. Luego para producir dicho flujo $\phi$, es necesario aplicar un campo magnético H, a través del circuito de excitación. El teorema de Ampère relaciona directamente el campo H a los amperios-vueltas magnetizantes del(os) circuito(s) de excitación, donde el campo H es proporcional a la fuerza magnetomotríz total aplicada.

De esta forma determinando la curva de saturación o vacío, a través de
mediciones de la tensión en los terminales de la armadura sin carga $U_{0}$
(aproximadamente igual a $E_{0}$) para diferentes valores de corriente de excitación $I_{exc}$, a velocidad de giro w constante, se obtiene una imagen de la curva de magnetización B=f(H).
\begin{figure}[H]
    \centering
    \includegraphics[scale=0.5]{recursos-Lab6/curvaDeVacioTeo.jpg}
    \caption{Curva de magnetización}
    \label{fig:curvaDeVacioTeo}
\end{figure}
\subsection{Funcionamiento como generador}
Cuando se trata de un generador de corriente continua, generalmente se
considera que gira a una velocidad constante e igual a su velocidad nominal. Así
las características externas de un generador mostrarán la evolución de una
cantidad eléctrica vs otra manteniendo constante la tercera.
Las tres variables eléctricas de un generador DC son:
\begin{itemize}
    \item $U_{c}$: Tensión en los terminales de la armadura.
    \item $I_{a}:$ Corriente de armadura.
    \item $I_{exc}:$ Corriente del campo de excitación principal.
\end{itemize}
\section{Lista de instrumentos}
\begin{table}[H]
	\caption{Lista de instrumentos de medición y componentes}
	\centering
	\begin{tabular}{|c|c|}
		\hline 
		Instrumento & Alcance ó especificaciones \\ \hline 
		Reostato &  (0-100) $\Omega$ \\  
		\hline 
		Osciloscopio &  - \\ 
		\hline 
		Transformador &  2 KVA; 2:1 \\  
		\hline 
		Voltímetro de hierro móvil &  (0-150) V\\  
		\hline 
		Amperímetro de hierro móvil &(0-1.2) A ó (0-6) A \\ 
		\hline
		Vatímetro & -	\\
		\hline
		Protecciones AC & 25 A; 380 V\\ 
		\hline
		Protecciones DC & -\\ 
		\hline
		Carga inductiva& (Se utilizo un motor DC) \\ 
		\hline
		Carga resistiva-capacitiva& 33 $\Omega$; 300 $\mu F$ \\ 
		\hline
		Puente rectificador & -  \\
		\hline
		Carga lineal & (200,400,600,800,1K)W  \\
		\hline
		Variac & 7 KVA @ 50 A 
		\\
		\hline 
	\end{tabular} 
\end{table}
\section{Condiciones de ensayo}
Estas son las precauciones y normativas necesarias para realizar el laboratorio de forma segura y efectiva: 
\begin{itemize}
    \item \textbf{Respecto a la medición de la curva en vacío:} La máquina a la que se le hara la prueba deberá estar conectada como generador. Dado que tiene que girar a velocidad constante, se utilizara algun mecanismo externo que brinde tales características. 
    
    Este sera un motor sincronico o una turbina. En el caso de emplearse un motor sincrónico el campo debera activarse al alcanzar una velocidad constante, debido a que para que exista par debe existir un desfasaje entre el campo magnético del rotor y el del estator.
    \item \textbf{Respecto a la medición de resistencias:} Se trabajara asegurandose que la corriente máxima alcanzada en $I_{f}$ ó $I_{a}$ sea menor o igual al 20 \% de la corriente nominal. La resistencia obtenida mediante la pendiente de la recta debe ajustarse de acuerdo a la temperatura.
    \item \textbf{Respecto a la medición de la característica de regulación (Motor):} Se debe mantener $U_{c}$ (tensión en los bornes) en su valor nominal y a velocidad nominal. 
    \item \textbf{Respecto a la  vestimenta:} No usar franelas o camisas manga larga, llevar zapatos de goma y pantalones. No usar collares ni pulseras de metal.
    \item \textbf{Previo a las pruebas:} Hacer primero el montaje antes de energizar, al culminarlo preguntar al profesor si las conexiones son correctas para proceder con las pruebas.
    \item \textbf{Respecto a la comunicación:} Mantener informado sobre cualquier cambio en el montaje al compañero de laboratorio y por sobre todo informar si el circuito se encuentra energizado o no.
	\item \textbf{Respecto a las curvas observadas:} No aceptar como adecuada una curva que este llena de ruido, ya que se puede deber a que algún elemento puede estar actuando como antena, esto originara imprecisión en los resultados.
	\item \textbf{Respecto al numero de mediciones:}
	Realizar al menos 5 mediciones para condiciones distintas.
	\item \textbf{Respecto a la elección de componentes y las conexiones:} Evitar los componentes que puedan funcionar como antenas (como resistencias de shunt de tipo mariposa u algun otro que se encuentre muy expuesto) y cuidar los contactos de cada conexión.
    \item \textbf{Respecto a la manipulación:} En caso de maniobrar el circuito energizado manipular con la mano derecha, buscando mayores probabilidades de sobrevivir en caso de un accidente eléctrico.
\end{itemize}
\section{Procedimiento}
\subsection{Características internas}
\begin{enumerate}
    \item Lo primero sera, hallar las resistencias internas de campo y armadura por lo que se realizaran las conexiones como en la figura \ref{fig:diagMedRes}.
    \item Se conectara unicamente el lado de campo con una fuente DC al valor nominal.
    \item Manteniendo la tensión $V_{f}$ fija y variando el reostato en pasos equidistantes se tomara nota de los valores de tensión y corriente. Se tomaran al menos 4 mediciones.
    \item Se desconectara la alimentación del lado de campo y se conectara la fuente del lado de armadura a tensión nominal.
    \item Manteniendo la tensión $V_{a}$ fija y desplazando la cuchilla de arranque en pasos equidistantes se tomara nota de los valores de tensión en los bornes y la tensión en la resistencia de shunt. Se tomaran al menos 5 mediciones.
    \item Se desconectara la alimentación del lado de armadura y se desenergizara el circuito.
    \item Se armara el esquema de conexiones conectando el motor DC como generador de acuerdo a la figura \ref{fig:diagMedCurvaSat}.
    \item Se encendera el generador sincronico y solo cuando alcance una velocidad constante se accionara el campo mediante las protecciones DC.
    \item Manteniendo la velocidad del eje a su valor nominal se variara el reostato del lado de campo de forma que los datos sean tomados en espacios apropiados que permitirán una exactitud de la curva graficada entre nula excitación y 125 \% de la tensión nominal ($U_{nom}$), en la parte lineal de la curva con incrementos de 20 \% de $U_{nom}$ y pasos del 10 \% de $U_{nom}$,
    alrededor del codo de saturación que suele estar entre el 80 \% y el 110 \% de la $U_{nom}$.  
    \item Se repetira el proceso del punto previo; pero desde el ultimo punto alcanzado hasta el valor minimo, respetando en lo posible que los pasos sean iguales. jajajajaja
\end{enumerate}
\subsection{Características externas}
\begin{enumerate}
    \item Se realiza la conexión de la fig \ref{fig:diagGeneradorIndependiente} de generador independiente.
    \item Una vez conectado todo el circuito con el motor que funciona como generador, empleando el tacometro se verifica velocidad nominal en el generador.
    \item Se ajusta $U$ inicalmente sin carga hasta alcanzar $U_{nom}$ con $I_{ext}$, una vez alcanzado $U_{nom}$ se registra el valor.
    \item Se conecta carga al sistema, generando un cambio en los valores de $U$ e $I_a$.
    \item Se ajusta $U$ con $I_{ext}$ hasta alcanzar valores nominales. Se registra el valor de $I_{ext}$ e $I_a$.
    \item Se varía la carga y se repite hasta llegar a valores nominales.
    \item Una vez registrado el ultimo valor se verifica que se encuentre en $U_{nom}$ e $I_{anom}$ en caso que sean diferentes se ajustarán a valores nominales, una vez verificado se desconectará la carga de forma gradual hasta retirarla completamente, luego se registra e valor de $U$ sin carga. 
\end{enumerate}
\subsection{Generador Shunt}
\begin{enumerate}
    \item Se realiza la conexión como el la fig \ref{fig:diagGeneradorShunt} de generador shunt.
    \item Para medir la curva característica de carga se ajusta $U_{nom}$ sin carga a travez de $I_{ext}$ y $N_{ctte}$ tomando como valor inicial $I_a = 0$.
    \item Se agrega carga al generador, se verifica que $I_{ext}$ se mantenga constante en caso contrario se ajusta con $R_p$.
    \item Al verificar los valores constante se toman las medidas de $U$ e $I_a$ se varia la carga y se repite el paso 3 y 4.
    \item Al tener los datos registrados se realiza la curva característica de regulación
    \item Se ajusta $U$ inicalmente sin carga hasta alcanzar $U_{nom}$ con $I_{ext}$, una vez alcanzado $U_{nom}$ se registra el valor.
    \item Se conecta carga al sistema, generando un cambio en los valores de $U$ e $I_a$.
    \item Se ajusta $U$ con $I_{ext}$ hasta alcanzar valores nominales. Se registra el valor de $I_{ext}$ e $I_a$.
    \item Se varía la carga y se repite hasta llegar a valores nominales.
    \item Una vez registrado el ultimo valor se verifica que se encuentre en $U_{nom}$ e $I_{anom}$ en caso que sean diferentes se ajustarán a valores nominales, una vez verificado se desconectará la carga de forma gradual hasta retirarla completamente, luego se registra e valor de $U$ sin carga. 
\end{enumerate}
\subsection{Motor independiente}
\begin{enumerate}
    \item Se realiza la conexión de la fig \ref{fig:diagMotorIndependiente} de motor independiente.
    \item Se realiza la medición de la curva electromecánica de velocidad manteniendo Uc constante e $I_{ext}$ constante.
    \item Se anotan las mediciones de $I_a$ y de la velocidad.
    \item Se varia Rs variando directamente $I_a$ se anota los valores y se repite hasta hacer varias mediciones.
\end{enumerate}
\section{Diagramas}
\begin{figure}[H]
    \centering
    \includegraphics[scale=0.5]{./recursos-Lab6/diagMedRes.png}
    \caption{Diagrama de conexión pruebas de resistencias internas}
    \label{fig:diagMedRes}
\end{figure}
\begin{figure}[H]
    \centering
    \includegraphics[scale=0.5]{./recursos-Lab6/diagMedCurvaVacio.png}
    \caption{Diagrama de conexión pruebas de Prueba curva de vacio}
    \label{fig:diagMedCurvaCaracteristica}
\end{figure}
\begin{figure}[H]
    \centering
    \includegraphics[scale=0.5]{./recursos-Lab6/diagGeneradorIndependiente.png}
    \caption{Diagrama de conexión de generador independiente}
    \label{fig:diagGeneradorIndependiente}
\end{figure}
\begin{figure}[H]
    \centering
    \includegraphics[scale=0.5]{./recursos-Lab6/diagGeneradorShunt.png}
    \caption{Diagrama de conexión de generador shunt}
    \label{fig:diagGeneradorShunt}
\end{figure}
\begin{figure}[H]
    \centering
    \includegraphics[scale=0.5]{./recursos-Lab6/diagMotorIndependient.png}
    \caption{Diagrama de conexión de motor independiente}
    \label{fig:diagMotorIndependiente}
\end{figure}
\section{Calculos preliminares}
\subsection{Características internas}
\subsubsection{Resistencia interna }
De acuerdo al diagrama de la figura \ref{fig:diagramaMedRes}, se espera una curva característica de tensión vs corriente para el lado de campo con una forma similar al de la figura \ref{fig:curvaCaractPreVILadoCampo} y en el lado de armadura su forma se asemejara a la figura \ref{fig:curvaCaractPreVILadoArmadura}.
\begin{figure}[H]
    \centering
    \includegraphics[scale=0.5]{./recursos-Lab6/curvaCaractPreVILadoCampo.jpg}
    \caption{Curva característica esperada V vs $I_{f}$}
    \label{fig:curvaCaractPreVILadoCampo}
\end{figure}
La pendiente de la curva previa proporcionara el valor de la resistencia de campo.
\begin{figure}[H]
    \centering
    \includegraphics[scale=0.5]{./recursos-Lab6/curvaCaractPreVILadoArmadura.jpg}
    \caption{Curva característica esperada V vs $I_{a}$}
    \label{fig:curvaCaractPreVILadoArmadura}
\end{figure} 
Como se puede observar en la figura \ref{fig:curvaCaractPreVILadoArmadura} a diferencia de la figura \ref{fig:curvaCaractPreVILadoCampo} no es una recta completamente. Al inicio de la curva se espera un comportamiento no lineal debido a la resistencia de las escobillas; Aunque desde cierto punto se vuelve lineal debido a que el ya mencionado efecto es despreciable, por lo que la pendiente de la zona lineal corresponde con el valor de la resistencia de armadura. 
\subsection{Características externas}

\section{Resultados experimentales}

\section{Resultados Teóricos}

\section{Análisis de resultados}
\section{Conclusión}
\newpage
\section{Anexos}

\end{document}
