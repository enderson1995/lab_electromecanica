\documentclass[11pt,letterpaper]{article}     % Tipo de documento y otras especificaciones
\usepackage[utf8]{inputenc}                   % Para escribir tildes y eñes
\usepackage[spanish]{babel}                   % Para que los títulos de figuras, tablas y otros estén en español
\usepackage[apaciteclassic]{apacite}
\usepackage{geometry}    
\usepackage{textcomp}
\geometry{left=25mm, right=25mm, top=25mm, bottom=25mm} % Tamaño del área de escritura de la página
\usepackage{amsmath}      % Los paquetes ams son desarrollados por la American Mathematical Society
\usepackage{amsfonts}     % y mejoran la escritura de fórmulas y símbolos matemáticos.
\usepackage{booktabs}
\usepackage{subfig}
\usepackage{amssymb}
\usepackage{graphicx}     % Para insertar gráficas
\usepackage{float}		% Para ubicar las tablas y figuras justo después del texto
\usepackage{pdfpages}
\batchmode
\usepackage{enumerate}
\usepackage{siunitx}
\pagestyle{plain} 
\usepackage{graphics}
\pagenumbering{arabic}
\usepackage{multicol}   % Para varias columnas
\usepackage{multirow}
\usepackage{color}%Paquete para colocar color al texto
%====================Español Venezolano Rápido============================
\renewcommand\tablename{Tabla}
\renewcommand\figurename{Figura}
%\renewcommand\prefacename{Prefacio}
\renewcommand\refname{REFERENCIAS}
%\renewcommand\bibname{REFERENCIAS}
\renewcommand\abstractname{Resumen}
%\renewcommand\chaptername{CAPÍTULO}
\renewcommand\appendixname{Apéndice}
\renewcommand\contentsname{ÍNDICE GENERAL}
\renewcommand\listfigurename{LISTA DE FIGURAS}
\renewcommand\listtablename{LISTA DE TABLAS}
\renewcommand\indexname{Índice Alfabético}
\renewcommand\partname{Parte}

%\renewcommand\enclname{Adjunto}
%\renewcommand\ccname{Copia a}
%\renewcommand\headtoname{A}
%\renewcommand\pagename{Página}
%\renewcommand\seename{véase}
%\renewcommand\alsoname{véase también}
%\renewcommand\proofname{Demostración}
%\renewcommand\glossaryname{Glosario}
%===================  Español venezolano =====================


\author{\\Omaña Enderson CI:  24.757.361 \\Raven Guillermo CI: 25.476.227\\Profesor: Crespo Jorge \vspace*{1in}}
\title{Universidad Central de Venezuela\\{ Facultad de Ingeniería\\Escuela de Ingeniería Eléctrica\\ Conversión Electromecánica de la energía\\\vspace*{1.5in} }Pre laboratorio 8\\MAQUINAS ASINCRÓNICAS O DE INDUCCIÓN\vspace*{1.35in}}
\date{Caracas, \today}

\begin{document}	% Inicio del documento
\maketitle							% Título
\newpage
\tableofcontents
\newpage
\section{Objetivos}
\begin{itemize}
	\item Presentar los métodos de medición y las condiciones de ensayos mínimas necesarias para la realización de las pruebas de laboratorio que son base para la determinación del circuito equivalente convencional.
	\item Deducir las ecuaciones matemáticas para la determinación de los parámetros del circuito equivalente.
	\item Determinar las curvas características más importantes de las máquinas de corriente continua.
    \item Presentar los algoritmos de los métodos iterativos para la determinación y ajuste del circuito equivalente, respectivamente.
    \item Comprobar a través de determinaciones, la validez de los métodos presentados para la obtención del circuito equivalente de una maquina de inducción.
\end{itemize}
\section{Marco Teórico}
\subsection{Método A}
Se usa para maquinas cuyo ensayo de rotor trabado fue realizado con tensiones de frecuencia menor al valor nominal, tal que: 
\begin{equation}
	f_{cc}\approx 25\%f_{nom}
\end{equation}
Estos son los pasos a seguir del método:
\begin{enumerate}
	\item Se deben conocer la tensión y corriente por fase, en el estátor consumida durante la prueba de vacío $V_{1,0}$ y $I_{1,0}$. La potencia por fase consumida por la maquina en vacío $P_{1,0}$. Tensión y corriente por fase, en el estátor consumida durante la prueba de rotor trabado, $V_{1,cc}$ y $I_{1,cc}$ y la potencia por fase consumida por la maquina durante el ensayo de rotor trabado, $P_{1,cc}$.
	\item Asumir un valor de $X_{1}$/$X_{2}$. En los casos que no se disponga de esta información tomar en cuenta los valores sugeridos en el estándar IEEE 112, segun el código ó ``Letra diseño" NEMA de la maquina:
	\begin{itemize}
		\item Diseño NEMA A,D y rotor bobinado: $X_{1}$/$X_{2}$ = 1.
		\item Diseño NEMA B: $X_{1}$/$X_{2}$ = 0,67.
		\item Diseño NEMA C: $X_{1}$/$X_{2}$ = 0,43.
	\end{itemize}
	\item Asumir un valor inicial de la reactancia de magnetización $X_{m,0}$ y estatórica de dispersión $X_{1,0}$. 
	
	Se toman como valores iniciales los valores obtenidos en las pruebas de vacío y rotor trabado de modo que quedarían las siguientes expresiones:
	\begin{align}
		X_{1_{(0)}} = \frac{\sqrt{\left(\frac{V_{1,cc}}{I_{1,cc}}\right)^{2}-\left(\frac{P_{1,cc}}{I_{1,cc}^{2}}\right)^{2}}}{1+\frac{X_{2}}{X_{1}}}\label{X10equ}\\
		X_{m_{(0)}} = \left\|\frac{V_{1,0}}{I_{1,0}}-(R_{1}+jX_{1,(i)})\right\| \label{Xm0equ}
	\end{align}
	\item Calcular $Q_{1,0}$ y $Q_{1,cc}$ respectivamente mediante:
	\begin{align}
		Q_{1,0} = \sqrt{(V_{1,0}\cdot I_{1,0})^{2}-P_{1,0}^{2}} \label{Q10equ}\\
		Q_{1,cc} = \sqrt{(V_{1,cc}\cdot I_{1,cc})^{2}-P_{1,cc}^{2}} \label{Q1ccequ}
	\end{align}
	\item Calcular $X_{m,(i+1)}$:
	\begin{equation}
		X_{m,(i+1)} = \frac{V_{1,0}^{2}}{Q_{1,0}-I_{1,0}^{2}\cdot X_{1,(i)}}\cdot \frac{1}{\left(1+\frac{X_{1,(i)}}{X_{m,(i)}}\right)^{2}}\label{Xm(i+1)equ}
	\end{equation}
	\item Calcular $X_{1,cc,(i)}$:
		\begin{equation}
	X_{1,cc,(i)} =  \frac{Q_{1,cc}}{I_{1,cc}^{2}\cdot\left(1+\frac{X_{1}}{X_{2}}+\frac{X_{1,(i)}}{X_{m,(i)}}\right)}\cdot\left(\frac{X_{1}}{X_{2}}+\frac{X_{1,(i)}}{X_{m,(i)}}\right) \label{X1cciequ} 
	\end{equation}
	\item Calcular $X_{1,(i+1)}$:
	\begin{equation}
		X_{1,(i+1)} = \frac{f_{nom}}{f_{cc}}X_{1,cc,(i)} \label{X1(i+1)equ}
	\end{equation}
	\item Para i = i+1 repetir paso 5 hasta el paso 7 hasta que los valores de las reactancias de dispersión y mutua se estabilicen alrededor de un 0,1\% de diferencia; es decir:
	\begin{align}
		|X_{1,(i+1)}-X_{1,(i)}| \leq 0,001 \label{esperadoX1}\\
		|X_{m,(i+1)}-X_{m,(i)}| \leq 0,001 \label{esperadoXm}
	\end{align}
	\item Calcular $X_{2}$:
	\begin{equation}
		X_{2} = \frac{X_{1,(i+1)}}{\frac{X_{1}}{X_{2}}} \label{X2equ}
	\end{equation}
	\item Determinar gráficamente y a partir de las mediciones realizadas el valor de las perdidas mecánicas $P_{mec}$.
	\item Calcular $P_{fe}$:
	\begin{equation}
		P_{fe} = P_{1,0}-\frac{P_{mec}}{3}-I_{1,0}^{2}\cdot R_{1}
	\end{equation}
	\item Calcular $g_{fe}$:
	\begin{equation}
		g_{fe} = \frac{P_{fe}}{V_{1,0}^{2}}\left(1+\frac{X_{1,(i+1)}}{X_{m,(i+1)}}\right)^{2} \label{gfeequ}
	\end{equation} 
	\item Calcular $R_{fe}$:
	\begin{equation}
	R_{fe} = \frac{1}{g_{fe}}
	\end{equation}
	\item Calcular $R_{2}$:
	\begin{equation}
		R_{2} = \left(\frac{P_{cc}}{I_{1,cc}^{2}}-R_{1}\right)\left(1+\frac{X_{2}}{X_{m,(i+1)}}\right)^{2}-\frac{X_{2}^{2}}{X_{1,(i+1)}^{2}}\cdot X_{1,cc,(i)}^{2}\cdot g_{fe}
	\end{equation}
\end{enumerate}

\section{Lista de instrumentos}
\begin{table}[H]
	\caption{Lista de instrumentos de medición y componentes}
	\centering
	\begin{tabular}{|c|c|}
		\hline 
		Instrumento & Alcance ó especificaciones \\ \hline 
		Vatimetro &  fp = 0,2 \\  
		\hline 
		Vatimetro &  fp Alto\\  
		\hline 
		Reostato &  (0-33) $\Omega$; 4.2 A \\  
		\hline
		Termometro ó Termocupla &  - \\  
		\hline
		Transformador de corriente &  - \\  
		\hline
		Reostato &  (0-100) $\Omega$; 2.4 A \\  
		\hline
		Voltímetros de bobina móvil y hierro móvil &  (0-150) V/ (0-15) V/(0-30) V/(0-300) V\\  
		\hline 
		Multímetro CEN-TECH & -\\  
		\hline 
		Resistencia de shunt & (En laboratorio se determinaran)\\  
		\hline
		Tacometro & -\\  
		\hline
		Carga lineal & 200 w, 400w, 800w, 1kw\\  
		\hline
		Amperímetro de bobina móvil y hierro móvil &(0-1.2) A/ (0-6) A/(0-30) A \\ 
		\hline
		Protecciones AC & 25 A; 380 V\\ 
		\hline
		Protecciones DC & -\\ 
		\hline
		\multirow{5}{3cm}{Motor AC} & - V \\ 
		\cline{2-2}
		& - A  \\
		\cline{2-2}
		& - rpm\\
		\cline{2-2}
		&- hp\\
		\cline{2-2}
		&carga -\%\\
		\hline
		\multirow{5}{3cm}{Generador DC} & 3 KV\\ 
		\cline{2-2}
		& 125 V \\
		\cline{2-2}
		&26,5 A\\
		\cline{2-2}
		& 1000 rpm\\
		\hline
		
		
	\end{tabular} 
\end{table}
\section{Condiciones de ensayo}
Estas son las precauciones y normativas necesarias para realizar el laboratorio de forma segura y efectiva: 
\begin{itemize}
    \item \textbf{Respecto a la prueba de vacío con rotor cortocircuitado:} La máquina a la que se le hará la prueba deberá estar conectada como motor. Es necesario tener especial cuidado de no seguir reduciendo la tensión cuando la maquina comience elevar la corriente, debido a que esto puede causar el colapso de la maquina.
   
    Es recomendable que los vatímetros posean un factor de potencia bajo.
    \item \textbf{Respecto a la medición de resistencia estatórica:} Se trabajara asegurándose que la corriente máxima alcanzada sea menor o igual al 10 \% de la corriente nominal. La resistencia obtenida deberá ajustarse de acuerdo a la temperatura.
    \item \textbf{Respecto a la prueba de rotor trabado:} Se debe tener especial cuidado en las cercanías de la carga completa, debido a las corrientes que se pueden alcanzar, por este motivo se trabajara a tensión reducida con un vatímetro preferiblemente de alto factor de potencia, se recomienda registrar  la temperatura del devanado del estátor o la resistencia del devanado del estátor.
    
    Se debe tener cuidado de no sobre calentar los devanados. Tomando las lecturas más altas primero y las lecturas más bajas en sucesión (Según IEEE Std 112-2004). Ayudará a igualar la temperatura.
    \item \textbf{Respecto a la medición de la curva de carga:} Se debe medir a velocidad constante.
    \item \textbf{Respecto a la  vestimenta:} No usar franelas o camisas manga larga, llevar zapatos de goma y pantalones. No usar collares ni pulseras de metal.
    \item \textbf{Previo a las pruebas:} Hacer primero el montaje antes de energizar, al culminarlo preguntar al profesor si las conexiones son correctas para proceder con las pruebas.
    \item \textbf{Respecto a la comunicación:} Mantener informado sobre cualquier cambio en el montaje al compañero de laboratorio y por sobre todo informar si el circuito se encuentra energizado o no.
	\item \textbf{Respecto a las curvas observadas:} No aceptar como adecuada una curva que este llena de ruido, ya que se puede deber a que algún elemento puede estar actuando como antena, esto originara incertidumbre en los resultados.
	\item \textbf{Respecto al numero de mediciones:}
	Realizar al menos 5 mediciones para condiciones distintas.
	\item \textbf{Respecto a la elección de componentes y las conexiones:} Evitar los componentes que puedan funcionar como antenas (como resistencias de shunt de tipo mariposa u algún otro que se encuentre muy expuesto) y cuidar los contactos de cada conexión.
    \item \textbf{Respecto a la manipulación:} En caso de maniobrar el circuito energizado manipular con la mano derecha, buscando mayores probabilidades de sobrevivir en caso de un accidente eléctrico.
\end{itemize}
\section{Procedimiento}
\subsection{Medición de resistencia estatórica}
\begin{enumerate}
	\item Lo primero sera, hallar las resistencias internas por fase, por lo que se realizaran las conexiones como en la Figura \ref{fig:diagMedResEstatorica}.
	\item Se alimentara las fases del estátor con tensión DC a una fracción de la tensión nominal.
	\item Manteniendo la tensión fija, se irán cambiando la resistencia del reostato de 100 $\Omega$, se tomara nota de los valores de tensión y corriente (en caso de ser muy elevada la corriente, se colocara una resistencia de shunt para realizar la medición) en cada fase. Se tomaran al menos 4 mediciones por cada fase.
	\item Se verificara si la conexión del estátor se encuentra en delta o en estrella, ya que de ser delta la resistencia por fase sera:
	\begin{equation}
	R_{fase} = \frac{V_{DC_{\phi}}}{2\cdot I_{DC_{\phi}}}
	\end{equation} 
	En delta sera:
	\begin{equation}
	R_{fase} = \frac{3\cdot V_{DC_{\phi}}}{2\cdot I_{DC_{\phi}}}
	\end{equation} 
	\item Se des-energizara el circuito.
\end{enumerate}
\subsection{Prueba en vacío con rotor cortocircuitado}
\begin{enumerate}
	\item Se armara el circuito presente en la Figura \ref{fig:diagPruebaDeVacio}, es importante denotar que para cada medición se tomara nota de la tensión  de linea, corriente de linea y potencia.
	\item Se iniciaran las mediciones desde donde rompa la inercia inicial el motor hasta la tensión nominal y se ira bajando en pasos equidistantes hasta que la corriente en vez de bajar comience a subir.
\end{enumerate}
\subsection{Prueba de rotor trabado}
\begin{enumerate}
	\item Se armara el circuito presente en la Figura \ref{fig:diagPruebaDeRotorTrabado}, es importante denotar que para cada medición se tomara nota de la tensión  de linea, corriente de linea y potencia.
	\item Se realizara al menos una medición mientras el rotor se encuentra trabado, se espera que la corriente sea la nominal y la potencia activa.
\end{enumerate}
\subsection{Efecto de carga}
\begin{enumerate}
	\item Se armara el circuito presente en la Figura \ref{fig:diagCarga}, es importante denotar que para cada medición se tomara nota de la tensión  de linea, corriente de linea y potencia.
	\item Se medirán los parámetros para distintas cargas conectadas al generador DC alimentado por el motor AC (al menos 4).
\end{enumerate}
\section{Diagramas}
(Lo haremos a mano y cuando le entreguemos el laboratorio los pasaremos a computadora)
\end{document}
